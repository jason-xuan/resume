% !TEX program = xelatex

\documentclass{resume}
%\usepackage{zh_CN-Adobefonts_external} % Simplified Chinese Support using external fonts (./fonts/zh_CN-Adobe/)
%\usepackage{zh_CN-Adobefonts_internal} % Simplified Chinese Support using system fonts

\begin{document}
\pagenumbering{gobble} % suppress displaying page number

\name{Jingxuan(Jason) Zhu}

\basicInfo{
  \email{jingxuan@wustl.com} \textperiodcentered\ 
  \phone{(314)255-7834} \textperiodcentered\ 
  \linkedin[jason]{https://www.linkedin.com/in/zhu-jason-71664a146/}
  \github[jason-xuan]{https://github.com/jason-xuan}
}

\section{\faGraduationCap\ Education}
\datedsubsection{\textbf{Washington University in St. Louis (WUSTL)}, MO, USA}{2018 -- present}
\textit{Master} in Computer Science (CS) GPA 3.9/4.0
\datedsubsection{\textbf{South China University of Technology (SCUT)}, Guangzhou, China}{2014 -- 2018}
\textit{B.S.} in Software Engineering (SE) GPA 3.52/4.0

\section{\faCogs\ Skills}
\begin{itemize}[parsep=0.5ex]
	\item Programming Languages: Python > C\# > Julia = matlab > C++ = Javascript = Java
	\item Frameworks: numpy, pandas, scrapy
	\item Platform: Linux
	\item Development: Flask, RESTApi
\end{itemize}

\section{\faUsers\ Experience}
\datedsubsection{\textbf{WUSTL} the HurdleDMR.jl package for NLP}{11/2018 - present}
\role{Julia Developer}{Research Assistant}
Brief introduction: Develop and maintain a machine learning package in Julia
\begin{itemize}
	\item Changed the time for standardizing matrix to improve the effiency
	%\item Replaced local cluster mode to multithreading to reduce runtime
	\item Refactoring test codes and saperate each test case in Julia
\end{itemize}

\datedsubsection{\textbf{SCUT} fuzzy lagged bi-clustering}{6/2017 - 8/2018 }
\role{Research Assistant}{The China Natural Science Foundation Project}
\begin{itemize}
  \item Studied on mining bi-cluster patterns with lag and fuzziness from a time series dataset
  \item Read the paper and made a presentation to teammates 
  \item Designed a biclustering algorithm for time-lagged and fuzzy datasets
  \item Used python and C++ to implement the algorithm and verified the accuracy and efficiency of the algorithm by artificial datasets and the pigeon flock dataset 
\end{itemize}

\datedsubsection{\textbf{Chinese Academy of Sciences} Summer Internship Camp}{8/2017}
\role{Summer Camp}{}
Brief introduction: Implemented a news search system based on Latent Dirichlet Allocation
\begin{itemize}
  \item Crawled the data with scrapy framework using python 
  \item Accelerated the data reading with protobuf and LMDB 
  \item Utilized the LDA model to establish the relationship among document-topic-word 
  \item Displayed the result with jupyter – notebook 
\end{itemize}

\datedsubsection{\textbf{Kmeans algorithm on Geographic Data on AWS spark }}{11/2018-12/2018}
\role{Spark cloud developer}{Course Projects}
\begin{itemize}
  \item Configured AWS cloud setting
  \item Run spark program on AWS spark cluster 
  \item Presented the result with jupyter notebook
  \item Earned 100/100 points for this project
\end{itemize}

%\datedsubsection{\textbf{ NETEASE CLOUD MUSIC CRAWLER}}{11/2015-12/2015 }
%\role{Python developer}{Course Projects}
%\begin{itemize}
%	\item Designed a web crawler using python, and used requests and beautifulsoup4 to collect the data of the Netease %Cloud Music website and stored the data into Database for statistics 
%	\item Added a header to camouflage browser and controlled the access frequency 
%	\item Used the cache, MongoDB to increase the data storage speed 
%\end{itemize}

% Reference Test
%\datedsubsection{\textbf{Paper Title\cite{zaharia2012resilient}}}{May. 2015}
%An xxx optimized for xxx\cite{verma2015large}
%\begin{itemize}
%  \item main contribution
%\end{itemize}


% \section{\faHeartO\ Honors and Awards}
% \datedline{\textit{\nth{1} Prize}, Award on xxx }{Jun. 2013}
% \datedline{Other awards}{2015}

%\section{\faInfo\ Miscellaneous}
%\begin{itemize}[parsep=0.5ex]
%  \item GitHub: https://github.com/jason-xuan
%  \item Languages: English - Fluent, Mandarin - Native speaker
%\end{itemize}

\end{document}
